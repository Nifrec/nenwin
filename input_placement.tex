Marbles are used to represent input data, which can for example be tabular data, an image, a stream of sound or video, etc.
From the algorithm's side,
this simply means that a new set of Marbles can be added to the simulation at each timestep.
However, it is required to map the input data (together with a designated \textit{input region}) to a set of Marbles first, by means of an function called an \textit{input placer}.
Such input placers could be defined in many different ways. 
For example, a datapoint could be used to generate a single Marble, in which the datum is mapped to an attribute of the Marble, such as the position, the mass, the velocity, etc..
It is also possible to use information multiple datapoints to create a single Marble.
Furthermore, it is also possible to define a trainable input placer, which can be optimized using machine learning techniques.

It was chosen to use an input placer (the "GridInputPlacer") that simply spreads the Marbles evenly in a two-dimensional grid over the input region (in order of the input vector, from left to right, top to bottom). This grid-placement generalizes to higher dimensions. The GridInputPlacer creates a Marble for each datapoint at a unique grid-point.
Note that this choice is rather arbitrary: as of current there are no arguments why aligning the Marbles in another pattern (such as on a circle) would be less suitable for AI applications.

Two variants of the GridInputPlacer were used:
\begin{itemize}
	\item MassInputPlacer: maps each datapoint to a Marble, whose mass is set to the value of the datapoint. The initial velocity is by default set to $\vec{0}$. When another velocity vector $\vec{v}$ is used, the notation MassInputPlacer$(\vec{v})$ will be used.
	\item VelInputPlacer: maps each datapoint to a Marble, whose velocity is set to a vector of which each element equals the value of the datapoint. It sets the mass of each Marble to 1.0.
\end{itemize}

All variants set, for each Marble, the remaining attributes as follows:
\begin{itemize}
	\item the radius of the threshold radius to 100.
	\item the acceleration to $\vec{0}$.
	\item the \texttt{marble\_stiffness} to 1.
	\item the \texttt{node\_stiffness} to 0.
	\item the \texttt{marble\_attraction} to 0
	\item the \texttt{node\_attraction} to 1.
\end{itemize}

