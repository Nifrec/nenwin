Neural networks, such as Multilayer Perceptrons, Recurrent Neural Networks (e.g. the GRU \cite{gru_original}), Convolutional Neural Networks (e.g. the famous AlexNet \cite{alexnet}) and many more, are a popular class of trainable models for Artificial-Intelligence and datamining applications. Neural networks are a series of interconnected functions consisting of an affine map followed by a nonlinear activation function, such as the hyperbolic tangent. In essence, they are a trainable nonlinear functions. One may object that Recurrent Neural Networks are no functions, as they have a state. But they can be seen as a function that takes it own output back as an additional argument: their computation remains the same.

Humans, however, are not regarded as functions (the behavioral approach of psychology, which studied humans and other animals as functions, has now mostly been deprecated \cite{matlin2016cognition}). Their behaviour can change over time. Given the exact same environment at a later moment in time, humans may behave differently than they did the previous time.

This work explores a new paradigm based on particle simulation, that is designed to capture this flexibility of behaviour better. It is based on classical Newtonian mechanics, so the interaction between particles can be compared to interactions between stars, planets and asteroids. A set of particles forms a model, and their \textit{initial} positions, \textit{initial} velocities and masses are their learnable parameters. Input data is represented by particles as well, and their movement is governed by gravitational interactions with the particles in the model. The eventual position of the input particles is used to determine the output. Because these input particles can also attract the particles of the model, the model can change its shape, which in turn changes its computations. Hence the algorithm encoded in the model can, in theory, change over time, without retraining. 

It should be noted that the new paradigm, to be called \textsc{Nenwin} from now onward, is designed for applications in which it either needs to act as an active agent, such as robotics or games, or needs to generate different output over time, such as music generation. For tasks such as image classification it does not provide any benefit over Neural Networks, and such tasks are only of interest for verification purposes.